\begin{itemize}
	\item SNe without U, u', J, H, and K data are removed from the dataset, then SNe are separated into II+IIP,IIn,Ib,Ic
	\item Run initial SNCOSMO fit to get initial time of peak, using bounds on $t_0$ of $\pm20$ days from first point
	\item Clip points more than 50 days from peak, repeat process and tighten clipping range if necessary for each SN until all time of peaks are reasonable to within $\pm 5$ days
	\item Select best color to use for extrapolation based on number of points, quality of fit, etc. 
	\item Run snsedextend program to get color tables after selecting color for each SN (set peak and have $\pm3$ days)
	\item Calculate and minimize Bayesian Information Criterion (BIC) to determine best fit polynomial order
	\item Create posterior predictive fits showing best fit to color curve measured data
\end{itemize}

% Please add the following required packages to your document preamble:
% \usepackage{booktabs}
% \usepackage{graphicx}
\renewcommand{\labelitemii}{-}


\subsection{Color Table Generation}
\begin{itemize} 
\item A list of SNe with U,u',J,H, or K and optical data are provided to the snsedextend package with the following dictionaries used for fitting:
     \begin{itemize} 
        \item SN:Redshift
        \item SN:Color to use for extrapolation
        \item SN:E(B-V)
        \item SN:Type
        \item SNCosmo Model:Type
     \end{itemize}
\item redshift, host $r_v$, milky way $r_v$, and milky way E(B-V) are set as constants for SNCOSMO fitting
\item Host E(B-V) is given bounds of $\pm1$, $t_0$ is given a bound of $\pm3$ days
\item Magnitude data are translated to flux for SNCOSMO fitting
\item All Optical (BVRg'r') data are fit using each model in SNCOSMO corresponding to the SN Type given, best model is taken
\item Best fit model is translated to time and magnitude vectors for band in color chosen for extrapolation (i.e. B for U-B), which are interpolated using a 1D interpolation
\item The time grid for the extrapolation band (i.e. U,u',J,H or K) is used to define magnitude values for the second color (i.e. B for U-B) so that we have UV or IR and optical data (i.e. U,B or r,J) values at the same epochs. They are then subtracted to get colors
\item Colors are de-reddened using coefficients from O'Donnell 94
\item Color table is generated (See table 2) from these data points
\end{itemize}

\begin{table}[H]
\centering
\resizebox{\textwidth}{!}{%
\begin{tabular}{@{}RCCCCCCCCCCCCCCCC@{}}
\toprule
\textbf{Days After Peak} & \textbf{U-B} & \textbf{U-B Error} & \textbf{U-V} & \textbf{U-V Error} & \textbf{V-J} & \textbf{V-J Error} & \textbf{V-H} & \textbf{V-H Error} & \textbf{V-K} & \textbf{V-K Error} & \textbf{B-J} & \textbf{B-J Error} & \textbf{B-H} & \textbf{B-H Error} & \textbf{B-K} & \textbf{B-K Error} \\ \midrule
-2.7134 & -- & -- & -0.80 & 0.05 & -- & -- & -- & -- & -- & -- & -- & -- & -- & -- & -- & -- \\
-2.7097 & -- & -- & -0.84 & 0.05 & -- & -- & -- & -- & -- & -- & -- & -- & -- & -- & -- & -- \\
1.2693 & -- & -- & -0.45 & 0.06 & -- & -- & -- & -- & -- & -- & -- & -- & -- & -- & -- & -- \\
1.2730 & -- & -- & -0.54 & 0.06 & -- & -- & -- & -- & -- & -- & -- & -- & -- & -- & -- & -- \\
1.3150 & -0.56 & 0.06 & -- & -- & -- & -- & -- & -- & -- & -- & -- & -- & -- & -- & -- & -- \\
1.9837 & -- & -- & -- & -- & -- & -- & -- & -- & 1.97 & 0.19 & -- & -- & -- & -- & -- & -- \\
1.9837 & -- & -- & -- & -- & -- & -- & 1.98 & 0.10 & -- & -- & -- & -- & -- & -- & -- & -- \\
1.9837 & -- & -- & -- & -- & 1.39 & 0.08 & -- & -- & -- & -- & -- & -- & -- & -- & -- & -- \\
2.3312 & -0.54 & 0.06 & -- & -- & -- & -- & -- & -- & -- & -- & -- & -- & -- & -- & -- & -- \\
2.4487 & -- & -- & -- & -- & -- & -- & -- & -- & -- & -- & 0.63 & 0.04 & -- & -- & -- & -- \\
&... &  &  &  &  &  &  &...  &  &  & & &  &  & ... &  \\
53.8437 & -- & -- & -- & -- & -- & -- & -- & -- & 2.25 & 0.03 & -- & -- & -- & -- & -- & -- \\
53.8437 & -- & -- & -- & -- & -- & -- & 2.12 & 0.06 & -- & -- & -- & -- & -- & -- & -- & -- \\
54.8337 & -- & -- & -- & -- & -- & -- & 2.05 & 0.04 & -- & -- & -- & -- & -- & -- & -- & -- \\
54.8337 & -- & -- & -- & -- & 1.68 & 0.04 & -- & -- & -- & -- & -- & -- & -- & -- & -- & -- \\
56.8337 & -- & -- & -- & -- & -- & -- & -- & -- & 2.19 & 0.03 & -- & -- & -- & -- & -- & -- \\
56.8337 & -- & -- & -- & -- & -- & -- & 2.03 & 0.03 & -- & -- & -- & -- & -- & -- & -- & -- \\
56.8337 & -- & -- & -- & -- & 1.70 & 0.02 & -- & -- & -- & -- & -- & -- & -- & -- & -- & -- \\
64.8237 & -- & -- & -- & -- & -- & -- & 1.88 & 0.07 & -- & -- & -- & -- & -- & -- & -- & -- \\
64.8237 & -- & -- & -- & -- & 1.63 & 0.02 & -- & -- & -- & -- & -- & -- & -- & -- & -- & -- \\
64.8237 & -- & -- & -- & -- & -- & -- & -- & -- & 2.01 & 0.26 & -- & -- & -- & -- & -- & -- \\ 
\end{tabular}%
}
\caption{Example of Type II color table (abbreviated) generated by the snsedextend package. A similar table is created for each SN type, which is then read and binned before being fit with a polynomial. For more information on how these color tables are used, see sections 4.2 and 5.1.}
\label{my-label}
\end{table}

\

\subsection{Polynomial Fits to Measured Color Curves}
\begin{itemize}
\item Description of BIC calculation and minimization
\item Description of Posterior Predictive Fitting 
\end{itemize}

\

\subsection{SED Extrapolation}
\begin{itemize}
\item Color curve(s) created/described in sections 4.1 and 4.2 is sent to SED extrapolation function
\item Color is calculated from color curve best fit polynomial for each epoch in SED
\item Color at each epoch is used to define a linear extrapolation into the UV or IR
\begin{itemize}
\item Iterative process is used to define the endpoints/slope of the extrapolation so that area under line is equal to the color at that epoch
\item This process was tested and verified by Rick Kessler
\end{itemize}
\item Endpoints were chosen based on UV and IR range that will be used (These are variables in the package that can be changed by a user):
\begin{itemize}
\item UVrightBound=4000
\item UVleftBound=1200
\item IRleftBound=9000
\item IRrightBound=55000
\end{itemize}
\item SED is extrapolated linearly such that it reaches zero at these endpoints
\end{itemize}
\

\subsection{SED Spectroscopic Features}
\begin{itemize}
\item
Find a set of observed or model UV / NIR spectra from objects within each SN class. 
\begin{itemize}
\item
Type Ib / Ic: We may have some nice composite spectra (or at least a few isolated spectra) from M. Modjaz's SNYU group
\item
Type II, IIn, II-P: We may have some from the CfA team.  Possibly from CSP.  Possibly some are available on the online SN data repositories like the Open Supernova Catalog
\end{itemize}
\item
Pick the best available spectra (highest S/N ratio)
\begin{itemize}
\item
For any observed spectra that are noisy, run a Savitzky-Golay smoothing filter over them. Should not be needed for template spectra, which are effectively noiseless.
\item
Flatten the spectra by fitting with a simple polynomial, as is done in SNID. The flattenened spectra should have a median flux values close to 0 for any wavelength bin.   
\end{itemize}
\item
* Add flattened spectra on to the linear extrapolations. This should effectively "paint on" the spectral features, while maintaining the broad-band colors to be the same as what was set with the current snsedextend procedure.
\end{itemize}